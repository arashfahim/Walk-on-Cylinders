\documentclass{article}
\usepackage{graphicx} % Required for inserting images
\usepackage{tikz}
\newcommand{\tikzmark}[1]{\tikz[baseline,remember picture] \coordinate (#1) {};}

\usepackage{float}
\usepackage{amsmath}
\usepackage{amssymb}
\usepackage{listings}
\usepackage{xcolor}
\lstdefinestyle{mypython}{
    language=Python,
   backgroundcolor=\color{white},   
    basicstyle=\ttfamily\footnotesize,
    keywordstyle=\color{blue}\bfseries,
    stringstyle=\color{orange},
    commentstyle=\color{green!50!black}\itshape,
    numberstyle=\tiny\color{gray},
    numbers=left,
    stepnumber=1,
    numbersep=5pt,
    showstringspaces=false,
    breaklines=true,
    frame=single,
    rulecolor=\color{gray},
}

\usepackage{algorithm}
\usepackage[noend]{algpseudocode}
\title{Power Series Stochastic Modeling}
\author{Ryan Avallone and Arash Fahim and Cruz Fuller and Richard Tairouz Aslam}
\date{May 2025}

\begin{document}

\maketitle

\section{Introduction}
Here is our project \ldots\\
First, we are going to find the eigenvalue and eigenfunction

\begin{figure}[ht!]
    \centering
    \includegraphics[width=0.75\linewidth]{Stochastic Modeling.jpeg}
    \caption{Stochastic Modeling}
    \label{fig:enter-label}
\end{figure}
\clearpage
\section{Find Eigenvalue and Eigenfunction in 2D}
\subsection{Separation of Variables}

Assume a solution of the form:
\[
u(t, r) = G(t)\phi(r)
\]
Then,
\[
\frac{\partial u}{\partial t} = G'(t)\phi(r)
\]
and the spatial part is:
\[
\frac{1}{2} \left( \phi''(r) + \frac{1}{r} \phi'(r) \right) = \text{(spatial operator)}
\]
Substitute into the original PDE:
\[
G'(t)\phi(r) = \frac{1}{2} \cdot G(t) \cdot \left( \phi''(r) + \frac{1}{r}\phi'(r) \right)
\]
Divide both sides by \( G(t)\phi(r) \):
\[
\frac{G'(t)}{G(t)} = \frac{1}{2} \cdot \frac{ \phi''(r) + \frac{1}{r}\phi'(r) }{ \phi(r) }
\]
Since the left side depends only on \( t \) and the right side only on \( r \), both sides must equal a constant, say \( -\lambda \). Thus:
\[
\frac{G'(t)}{G(t)} = -\lambda, \quad \text{and} \quad \frac{1}{2} \cdot \frac{ \phi''(r) + \frac{1}{r}\phi'(r) }{ \phi(r) } = -\lambda
\]
This gives two ODEs:
\begin{align*}
G'(t) + \lambda G(t) &= 0 \\
\phi''(r) + \frac{1}{r}\phi'(r) + 2\lambda \phi(r) &= 0
\end{align*}
Solving the time equation:
\[
G(t) = C_1 e^{-\lambda t}
\]
Solving the Bessel Differential Equation
\[
\phi''(r) + \frac{1}{r}\phi'(r) + 2\lambda \phi(r) = 0
\]
Let \(\mu^2 = 2\lambda r^2\), then the spatial equation becomes:
\[
\phi''(r) + \frac{1}{r} \phi'(r) + \mu^2 \phi(r) = 0
\]
We will follow the method of 
Frobenius and look for the solution as a power series
\[
y = x^s\sum_{n=0}^{\infty}a_nx^n
\]
\[
y'= (n+s)\sum_{n=0}^{\infty}a_nx^{n+s-1}
\]
\[
y''=(n+s)(n+s-1)\sum_{n=0}^{\infty}a_nx^{n+s-2}
\]
We will substitute the series back into the Bessel equation
\[
r^2[(n+s)(n+s-1)\sum_{n=0}^{\infty}a_nx^{n+s-2}]
+ r[(n+s)\sum_{n=0}^{\infty}a_nx^{n+s-1}]
+\sum_{n=0}^{\infty}\mu^2a_nx^{n+s}
\]
Next we will combine similar sums:
\[
\sum_{n=0}^\infty[((n+s)^2-\mu^2)a_n + a_{n-2}] r^{n+s}
\]
Indicial Equation (from \( n = 0 \)):
\[
[(s^2 - \mu^2) a_0] r^s = 0 \quad \Rightarrow \quad s^2 = \mu^2 \Rightarrow s = \pm\mu
\]
So the roots are \( s = \mu \) and \( s = -\mu \).\\
For \( n = 1 \):
\[
[(1 + s)^2 - \mu^2] a_1 = 0
\]
For \( n=2\):
\[
a_2 = -\frac{a_0}{(2 + \mu)^2 - \mu^2}
= -\frac{a_0}{4 + 4\mu}
= -\frac{a_0}{4(1 + \mu)}
\]
For \( n=4\):
\[
a_4 = -\frac{a_2}{(4 + \mu)^2 - \mu^2}
= -\frac{a_2}{16 + 8\mu}
= \frac{a_0}{32(1 + \mu)(2 + \mu)}
\]
For \( n=6\):
\[
a_6 = -\frac{a_4}{(6 + \mu)^2 - \mu^2}
= -\frac{a_4}{36 + 12\mu}
= -\frac{a_0}{384(1 + \mu)(2 + \mu)(3 + \mu)}
\]
Thus, the series solution becomes:
let \(k=2n\):
\[
a_{2k} = \frac{(-1)^k a_0}{2^{2k}k!(1 + \mu)(2 + \mu) \cdots (k + \mu)}
\]
Choosing \(a_0=\frac{1}{2^\mu\Gamma(\mu+1)}\):
\[
J_0(r) = \sum_{k=0}^{\infty} \frac{(-1)^k}{(k!)^2} \left( \frac{r}{2} \right)^{2k}
\]
The Bessel equation solution is:
\[
\phi(r) = A J_0(\mu r)
\]
The general solution is
\[
u(t, r) = A e^{-\frac{\mu^2}{2} t} J_0(\mu r)
\]
\subsection{Applying the Boundary Condition for 2D }

Plugging in the boundary condition \( u(t, R) = 0 \), we get:
\[
u(t, R) = G(t)\phi(R) = 0 \quad \Rightarrow \quad \phi(R) = 0
\]
\[
\phi(R) = A J_0(\mu R) = 0 \quad \Rightarrow \quad J_0(\mu R) = 0
\]
where \(\mu = \sqrt{2\lambda}\).
Then, the general solution 
\[
u(t, r) = A e^{-\frac{\mu^2}{2} t} J_0(\mu r)
\]

Since \(J_0(\sqrt{2\lambda} R) = 0\), it follows that
\[
\sqrt{2\lambda} R = Z,
\]
where \(Z_n\) is the $n$th positive zero of \(J_0\). Thus,
\[
\lambda = \frac{1}{2} \left( \frac{Z}{R} \right)^2
\]
then,
\[
\lambda_n = \frac{1}{2} \left( \frac{Z_n}{R}\right)^2
\]
\subsection{Genearl Soulction for 2D Heat Equation}

Thus, the general solution can be written as a Bessel series:
\[
u(t, r) = \sum_{n=1}^\infty A_n e^{- \lambda_n t} J_0\left( \frac{Z_n}{R} r \right),
\]
where:
\[
\lambda_n = \frac{1}{2} \left( \frac{Z_n}{R} \right)^2,
\]
and \( Z_n \) is the \( n \)-th zero of the Bessel function \( J_0 \).

{\color{red}Arash: Overall, it looks correct. I should check the details more carefully later. I suppose the next step is to find the coefficients $A_n$ using the formula that Richard showed me in the last meeting.}

\subsection{Finding the Coefficients \(A_n\)}

We are given the initial condition:
\[
u(0, r) = f(r) = 1.
\]

{\color{red}Arash: 
Recall that with $u(0, r) = f(r) = 1$ the solution $u(t,r)$ correspond with the probability that Brownian motion does not hit the lateral boundary at time $T$. To familiarize you with the notation, we can express this in mathematical language by
{\Large
\[
u(T,r)=\tikzmark{P}\mathbb{P}\left(\max_{t\in[0,T]}|B_t|\tikzmark{maxdist}<R\Big||B_0|\tikzmark{given}=r\right)
\]
\begin{tikzpicture}[overlay,remember picture]
    \node (Pr) [below of = P, node distance = 3 em, anchor=south]{\large \textsf{Probability}};
    \draw[<-, in=180, out=-90] (P.south)++(.25em,-.5ex) to (Pr.west);

    \node (max) [below of = P, node distance = 4 em, anchor=west] {\large \textsf{Maximum distance to center $<R$}};
    \draw[<-, in=180, out=-90] (maxdist.south)++(.25em,-.5ex) to (max.west);
    
    \node (cond) [below of = P, node distance = 5 em, anchor=west] {\large \textsf{Given $B_0$ distance to center $=r$}};
    \draw[<-, in=0, out=-90] (given.south)++(.75em,-.5ex) to (cond.east);
\end{tikzpicture}
}
\vspace{2cm}

One possible modification of the initial value is
\[
u(0, r) = f(r) = 
\begin{cases}
    1& r\le \rho\\
    0&r>\rho
\end{cases}
\]
In this case, the solution and its interpretation are different:
{\Large 
\begin{equation}\label{eqn:cdf_B_T}
u(T,r,R,\rho)= \mathbb{P}\left(\max_{t\in[0,T]}|B_t| < R \text{ and } |B_T| \tikzmark{cdf}\le  \rho \Big||B_0|=r\right)
\end{equation}
\begin{tikzpicture}[overlay,remember picture]
    \node (cdfrho) [below of = cdf, node distance = 4 em, anchor=south]{\large \textsf{Distance of $B_T$ to center $\le\rho$}};
    \draw[<-, in=180, out=-90] (cdf.south)++(.25em,-.5ex) to (cdfrho.west);

\end{tikzpicture}
}
\vspace{2cm}

In other world, $u(T,r)$ gives the probability that the Brownian motion survives hitting boundary until time $T$ and, at time $T$, its distance to center of sphere does not exceed $\rho$. If we introduce $\rho$ as a variable, $u(T,r,\rho)$ is the \emph{cumulative probability distribution} of $|B_T|$ when $B$ survives hitting the boundary. When cumulative distribution is given, generating random samples according to distribution is simple.

}

\textbf{Step 1: Write the general solution at \(t = 0\)}  
\[
u(0, r) = \sum_{n=1}^\infty A_n J_0\left( \frac{Z_n}{R} r \right)
\]

\textbf{Step 2: Multiply both sides by } \( r J_0\left( \frac{Z_m}{R} r \right) \) \textbf{ and integrate from } \( 0 \) \textbf{ to } \( R \):
\[
\int_0^R 1 \cdot r J_0\left( \frac{Z_m}{R} r \right) \, dr
= \sum_{n=1}^\infty A_n \int_0^R r J_0\left( \frac{Z_n}{R} r \right) J_0\left( \frac{Z_m}{R} r \right) \, dr
\]

\textbf{Step 3: Use orthogonality of Bessel functions:}  
\[
\int_0^R r J_0\left( \frac{Z_n}{R} r \right) J_0\left( \frac{Z_m}{R} r \right) \, dr
= 0 \quad \text{if } n \ne m
\]
\[
\int_0^R r J_0\left( \frac{Z_n}{R} r \right) J_0\left( \frac{Z_n}{R} r \right) \, dr
= \frac{R^2}{2}[J_1(Z_n)]^2 \quad \text{if } n = m
\]

\textbf{Step 4: Solve for } \( A_n \):
\[
A_n = \frac{
\int_0^R r \cdot u(0, r) \cdot J_0\left( \frac{Z_n}{R} r \right) \, dr
}{
\int_0^R r \cdot \left[J_0\left( \frac{Z_n}{R} r \right)\right]^2 \, dr
}
= \frac{
\int_0^R r \cdot J_0\left( \frac{Z_n}{R} r \right) \, dr
}{
\int_0^R r \cdot \left[J_0\left( \frac{Z_n}{R} r \right)\right]^2 \, dr
}.
\]
\[
A_n = \frac{2}{R^2 [J_1(Z_n)]^2} \int_0^R r \cdot J_0\left( \frac{Z_n}{R} r \right) \, dr
\]

\textbf{Step 5: Use known integral:}
\[
\int_0^R r J_0\left( \frac{Z_n}{R} r \right) \, dr = \frac{R^2}{Z_n} J_1(Z_n)
\]

\textbf{Therefore:}
\[
A_n = \frac{2}{R^2 [J_1(Z_n)]^2} \cdot \frac{R^2}{Z_n} J_1(Z_n)
= \frac{2}{Z_n J_1(Z_n)}
\]

\[
\boxed{
A_n = \frac{2}{Z_n J_1(Z_n)}
}
\]

\textbf{Hence:}
\[
u(t, r) = \sum_{n=1}^{\infty}\left(\frac{2}{Z_nJ_1(Z_n)} \right)
e^{- \frac{1}{2} \left( \frac{Z_n}{R} \right)^2 t} \cdot J_0\left( \frac{Z_n}{R} r \right),
\]
\begin{figure}
    \centering
    \includegraphics[width=0.5\linewidth]{Cylindrical Bessel Equation.png}
    \caption{Bessel Functions $J_0\left(\frac{{Z_{{{n}}}}}{{R}} r\right)$ on $[0,1]$}
    \label{fig:enter-label-2}
\end{figure}
{\color{red}Arash: find out how to evaluate roots of Bessel functions in Python.}
\newpage
\section{Roots of Bessel Functions in Python}
\begin{lstlisting}[style = mypython]
import numpy as np
import matplotlib.pyplot as plt
from scipy.special import jv
from scipy.optimize import brentq

def bessel_Roots_for_dimension(n, num_roots=6):
    if n <= 2:
        print("Dimension must be at least 2.")
    # V = (n - 2)/2
    order = (n - 2) / 2
    x = np.linspace(0.01, 10*np.pi, 1000)
    y = jv(order, x)

    # Find roots
    roots = []
    step = 0.1
    x0 = 0.1
    while x0 < 30 and len(roots) < num_roots:
        if jv(order, x0) * jv(order, x0 + step) < 0:
           root = brentq(lambda x: jv(order, x), x0, x0 + step)
           roots.append(root)
        x0 += step
    return roots

bessel_Roots_for_dimension(2, 6)
\end{lstlisting}
\newpage
\section{The Heat Equation in Sphere Coordinates}
We know that the Sphere Coordinates are:
\begin{itemize}
    \item \(x=r\sin\theta \cos\phi\)
    \item \(y=r\sin\theta \sin\phi\)
    \item \(z=\cos\theta\)
\end{itemize}
Recall that our heat equation:
\[
\frac{\partial u}{\partial t}
=\frac{1}{2}\Delta u
= \frac{1}{2} \left(
\frac{\partial u^2}{\partial r^2}+
\frac{1}{r} \frac{\partial u}{\partial t}+
\frac{1}{r^2} \frac{\partial u^2}{\partial \theta}
\right)
\]
Now we are going to change into sphere coordinate:
\begin{align*}
\frac{\partial u}{\partial t}
  &= \frac{1}{2} \nabla ^2u \\
  &= \frac{1}{2} \left[
      \frac{1}{r^2} \frac{\partial}{\partial r} \left( r^2 \frac{\partial u}{\partial r} \right)
    + \frac{1}{r^2 \sin\theta} \frac{\partial}{\partial \theta} \left( \sin\theta \frac{\partial u}{\partial \theta} \right)
    + \frac{1}{r^2 \sin^2\theta} \frac{\partial^2 u}{\partial \phi^2}
    \right]
\end{align*}
{\color{orange}I have question regrading if u depend on \(\theta\) and \(\phi\) if not will the equation will be decrease into 1-D Heat Equation.}\\
{\color{red}Arash: If the initial condition, $u(0,r,\theta,\phi)=f(r,\theta,\phi)$, is independent of $\theta$ and $\phi$, i.e.,  $u(0,r,\theta,\phi)=f(r)$, then $u(t,r,\theta,\phi)=u(t,r)$. In this case, it is not a classical 1-D heat equation, $u_t=\frac12u_{xx}$. It is a different equation described by $u_t=\frac{1}{2r^2} \frac{\partial}{\partial r} \left( r^2 \frac{\partial u}{\partial r} \right)$}

\subsection{Assumption: Spherical Symmetry}

Assume $u(t, r, \theta, \phi) = u(t, r)$, i.e., $u$ is independent of $\theta$ and $\phi$. Then the Laplacian simplifies to its radial part:
\[
\Delta u = \frac{1}{r^2} \frac{\partial}{\partial r} \left( r^2 \frac{\partial u}{\partial r} \right)
\]

Thus, the heat equation becomes:
\[
\frac{\partial u}{\partial t} = \frac{1}{2r^2} \frac{\partial}{\partial r} \left( r^2 \frac{\partial u}{\partial r} \right)
\]

By the section above we know that
\(u(t,r)=G(t)\phi(r)\)

Also we know that:
\[
\frac{\partial u}{\partial t} = G'(t)\phi(r)
\]

The right hand side is
\[
Rhs=\frac{G(t)}{2r^2}\left(2r\phi'(r)+r^2\phi''(r)\right)
\]
\[
Rhs = G(t) \cdot\left(\frac{\phi'(r)}{r}+\frac{1}{2}\phi''(r)\right)
\]


Now substitute into the original PDE:
\[
 G'(t)\phi(r) = G(t) \cdot\left(\frac{\phi'(r)}{r}+\frac{1}{2}\phi''(r)\right)
\]

Divide both side by the \(G(t)\phi(r)\):
\[
\frac{G'(t)}{G(t)}=\frac{1}{\phi(r)}\left(\frac{\phi'(r)}{r}+\frac{1}{2}\phi''(r)\right)
\]
Once again we will set the two ODE to a constant \(-\lambda\):
\[
\frac{G'(t)}{G(t)}=\frac{1}{\phi(r)}\left(\frac{\phi'(r)}{r}+\frac{1}{2}\phi''(r)\right) = -\lambda
\]
For the time equation, it is the same for the 2D case from above:
\[
G(t) = C_1e^{-\lambda t}
\]

Solving the 3D Bessel Differential Equation:
\begin{align*}
\frac{1}{\phi(r)}\left( \frac{\phi'(r)}{r} + \frac{1}{2} \phi''(r) \right)
&= -\lambda \\
\frac{\phi'(r)}{r} + \frac{1}{2} \phi''(r)
&= -\lambda \phi(r)\\
2 \cdot \left( \frac{\phi'(r)}{r} + \frac{1}{2} \phi''(r) \right)
&= -2\lambda \phi(r) \\
\frac{2}{r} \phi'(r) + \phi''(r)
&= -2\lambda \phi(r) \\
\end{align*}
\[
\boxed{ \phi''(r) + \frac{2}{r} \phi'(r) + 2\lambda \phi(r) = 0 }
\]

Once again, Let \(\mu^2 = 2\lambda\), then the eqaution becomes:
\[
\phi''(r) + \frac{2}{r} \phi'(r) + \mu^2 = 0
\]

\subsection{Change of vairable to elmilnate the \(\frac{2}{r}\phi'(r)\)}:
\begin{align*}
\text{Let }:
\varphi&=\frac{v(r)}{r}\\ \varphi'(r)&=\frac{v'(r)\cdot r- v(r)}{r^2}\\
\varphi''(r) &=\frac{v'(r)}{r}-\frac{2v'(r)}{r^2}+\frac{2v(r)}{r^3}
\end{align*}
Plug then back to the spatial equation:
\begin{align*}
    \phi''(r)+\frac{2}{r}\phi'(r)&=
    \left[
    \frac{v'(r)}{r}-\frac{2v'(r)}{r^2}+\frac{2v(r)}{r^3}
    \right]+
    \left[
    \frac{v'(r)\cdot r- v(r)}{r^2}
    \right]\\
    &= \frac{v''(r)}{r}\\
    \Rightarrow \frac{v''(r)}{r}+\mu^2\frac{v(r)}{r}=0
\end{align*}

\begin{align*}
    v(r)&=A\sin(\mu r)+B\cos(\mu r)\\
        \\
        &=\frac{A\sin(\mu r)+B\cos(\mu r)}{r}
\end{align*}
Then the solution for spatial equation:
\boxed{\phi(r)=\frac{A\sin(\mu r)+B\cos(\mu r)}{r}}
\subsection{Boundary Condition for the 3D Case}

Applying the boundary condition \( u(t, R) = 0 \) to the separated solution \( u(t, r) = G(t)\phi(r) \), we obtain:
\[
u(t, R) = G(t)\phi(R) = 0
\quad \Rightarrow \quad \phi(R) = 0.
\]

In the 3D case, the transformed radial equation reduces to:
\[
\phi''(r) + \frac{2}{r} \phi'(r) + \mu^2 \phi(r) = 0,
\]
which is the spherical Bessel differential equation of order zero. The general solution is:
\[
\phi(r) = A \cdot \frac{\sin(\mu r)}{r} + B \cdot \frac{\cos(\mu r)}{r}.
\]

To ensure regularity at \( r = 0 \), we set \( B = 0 \) (since \( \frac{\cos(\mu r)}{r} \) diverges as \( r \to 0 \)). Thus,
\[
\phi(r) = A \cdot \frac{\sin(\mu r)}{r}.
\]

This function is precisely the \textbf{spherical Bessel function of the first kind of order zero} denoted \( j_0(\mu r) \):
\[
j_0(\mu r) = \frac{\sin(\mu r)}{r}.
\]

Therefore, we can write:
\[
\phi(r) = A \cdot j_0(\mu r),
\]
and the boundary condition \( \phi(R) = 0 \) gives:
\[
\sin(\mu R) = 0
\quad \Rightarrow \quad \mu R = n\pi, \quad n = 1, 2, 3, \dots
\]
which implies:
\[
\mu_n = \frac{n\pi}{R},
\quad \text{and} \quad \lambda_n = \frac{\mu_n^2}{2} = \frac{n^2 \pi^2}{2R^2}.
\]   
\subsection{Final Solution For 3D Heat Equation}

The general solution is:
\[
u_n(t, r) = A_n \cdot e^{-\lambda_n t} \cdot \frac{\sin\left( \frac{n\pi r}{R} \right)}{r}
\]

So the complete solution (for general initial conditions) is:
\[
u(t, r) = \sum_{n=1}^{\infty} A_n \cdot e^{ -\frac{n^2 \pi^2}{2R^2} t } \cdot \frac{ \sin\left( \frac{n\pi r}{R} \right) }{r}
\]


\subsection{Use Initial Condition to Find Coefficient}

We are given the initial condition:
\[
u(0, r) = f(r) = 1
\]

From the separated solution:
\[
u(t, r) = \sum_{n=1}^\infty A_n \cdot e^{-\lambda_n t} \cdot \frac{\sin\left( \frac{n\pi r}{R} \right)}{r}
\]

At \( t = 0 \), this becomes:
\[
u(0, r) = \sum_{n=1}^\infty A_n \cdot \frac{\sin\left( \frac{n\pi r}{R} \right)}{r} = 1
\]
For any index integer m, independent of the dummy index n\\
\[
f(r)\sin\frac{m\pi r}{r}=\sum_{n=1}^{\infty}A_n\sin\frac{n\pi r}{r}\sin\frac{m\pi r}{r}
\]
\[
\int_{0}^{R}f(r)\sin(\frac{m\pi r}{r})=\sum_{n=1}^{\infty}A_n\int_{0}^{R}\sin\frac{n\pi r}{r}\sin\frac{m\pi r}{r}
\]
\subsection{Orthogonality of Sine Functions (with weight \( r ^2\))}

We consider the inner product of two basis functions:
\[
\langle \phi_n, \phi_m \rangle = \int_0^R \phi_n(r) \cdot \phi_m(r) \cdot r^2 \, dr
= \int_0^R r^2\cdot\sin\left( \frac{n\pi r}{R} \right) \sin\left( \frac{m\pi r}{R} \right) dr
\]

Using the identity:
\[
\sin A \sin B = \frac{1}{2} \left[ \cos(A - B) - \cos(A + B) \right],
\]
we write:
\[
\int_0^R \sin\left( \frac{n\pi r}{R} \right) \sin\left( \frac{m\pi r}{R} \right) r^2\cdot dr
= \frac{1}{2} \left[ \int_0^R \cos\left( \frac{(n - m)\pi r}{R} \right) r^2\cdot dr - \int_0^R r \cos\left( \frac{(n + m)\pi r}{R} \right) r^2\cdot dr \right]
\]

Each of these integrals evaluates to zero when \( n \ne m \), due to symmetry over the interval \( [0, R] \). Thus:
\[
\boxed{
\int_0^R \sin\left( \frac{n\pi r}{R} \right) \sin\left( \frac{m\pi r}{R} \right)r^2 \cdot  dr = 0 \quad \text{for } n \ne m
}
\]

---

\subsection*{Norm of Each Basis Function (\( n = m \))}

When \( n = m \), we compute:
\[
\int_0^R r \cdot\sin^2\left( \frac{n\pi r}{R} \right)dr
\]

Using the identity:
\[
\sin^2 x = \frac{1 - \cos(2x)}{2},
\]
we find:
\[
\int_0^R r\cdot\sin^2\left( \frac{n\pi r}{R} \right)  dr = \frac{R^2}{2}
\quad \text{(standard result)}
\]

So the norm is:
\[
\boxed{
\int_0^R \left[ \frac{\sin\left( \frac{n\pi r}{R} \right)}{r} \right]^2 r\cdot dr = \int_0^R r\cdot\sin^2\left( \frac{n\pi r}{R} \right) dr = \frac{R^2}{2}
}
\]


\subsection*{Numerator of \( A_n \)}

We want to compute:
\[
\int_0^R f(r) \cdot \sin\left( \frac{n\pi r}{R} \right) \cdot r^2 \, dr
\quad \text{with } f(r) = 1
\]

We make the substitution:
\[
x = \frac{n\pi r}{R} \Rightarrow r = \frac{Rx}{n\pi}, \quad dr = \frac{R}{n\pi} dx
\]

When \( r = 0 \), \( x = 0 \); when \( r = R \), \( x = n\pi \)

So the integral becomes:
\[
\int_0^R r \cdot\sin\left( \frac{n\pi r}{R} \right) \, dr
= \frac{R^2}{n\pi} \int_0^{n\pi} x \sin(x) \, dx
\]

Now compute the integral \( \int x \sin x \, dx \) using integration by parts:

Let \( u = x \), \( dv = \sin x \, dx \), then \( du = dx \), \( v = -\cos x \)

\[
\int x \sin x \, dx = -x \cos x + \int \cos x \, dx = -x \cos x + \sin x
\]

Evaluate from \( 0 \) to \( n\pi \):

\[
\int_0^{n\pi} x \sin x \, dx = \left[ -x \cos x + \sin x \right]_0^{n\pi}
= -n\pi \cos(n\pi) + \sin(n\pi) - 0 = n\pi (-1)^{n+1}
\]

So the numerator becomes:
\[
\frac{R^2}{n\pi} \cdot n\pi (-1)^{n+1} = R^2 (-1)^{n+1}
\]

\[
\boxed{
\int_0^R r \sin\left( \frac{n\pi r}{R} \right) \, dr = R^2 (-1)^{n+1}
}
\]

---

\subsection{The General Solution for 3D heat Equation}

Putting everything together:
\[
A_n = \frac{R^2 (-1)^{n+1}}{R ^2/ 2} = 2(-1)^{n+1}
\]

So the full solution becomes:
\[
u(t, r) = \sum_{n=1}^{\infty} 2(-1)^{n+1} \cdot e^{- \frac{n^2 \pi^2}{2R^2} t} \cdot \frac{\sin\left( \frac{n\pi r}{R} \right)}{r}
\]
\begin{figure}
    \centering
    \includegraphics[width=0.5\linewidth]{Spherical Bessel Function.png}
    \caption{Spherical Bessel Function of\(j_0\): $\frac{\sin\left(\frac{Z_n}{R} r\right)}{r}$}
    \label{fig:l}
\end{figure}

\section{Extension to \(n\)-Dimensional Radial Case}

We want to eliminate the first derivative from a radial differential equation using a substitution.

\subsection*{Step 1: Change of Variable}

Let
\[
\phi(r) = r^{-\frac{n-2}{2}} v(r)
\]
Define
\[
\nu = \frac{n - 2}{2}
\]
Then,
\[
\phi(r) = r^{-\nu} v(r)
\]

\subsection*{Step 2: Compute Derivatives}

We compute the first and second derivatives of $\phi(r)$:
\[
\phi'(r) = -\nu r^{-\nu - 1} v(r) + r^{-\nu} v'(r)
\]
\[
\phi''(r) = \nu(\nu + 1) r^{-\nu - 2} v(r) - 2\nu r^{-\nu - 1} v'(r) + r^{-\nu} v''(r)
\]

\subsection*{Step 3: Substitution into the Original Equation}

Suppose the original radial equation for \(\phi(r)\) is
\[
\phi''(r) + \frac{n-1}{r} \phi'(r) + \mu^2 \phi(r) = 0.
\]
where \(n\) is the dimensions 
Substitute
\[
\phi(r) = r^{-\nu} v(r), \quad \text{where } \nu = \frac{n-2}{2}.
\]

Using the derivatives computed in Step 2,\\
Plugging these into the original equation:
\[
\begin{aligned}
& \phi''(r) + \frac{n-1}{r} \phi'(r) + \mu^2 \phi(r) \\
=\, & \left[ \nu(\nu+1) r^{-\nu - 2} v(r) - 2\nu r^{-\nu - 1} v'(r) + r^{-\nu} v''(r) \right] \\
& + \frac{n-1}{r} \left[-\nu r^{-\nu - 1} v(r) + r^{-\nu} v'(r)\right] + \mu^2 r^{-\nu} v(r).
\end{aligned}
\]

Simplify terms by factoring powers of \(r\):
\[
\begin{aligned}
=\, & r^{-\nu - 2} \left[ \nu(\nu+1) v(r) - 2\nu r v'(r) + r^{2} v''(r) \right] \\
& + r^{-\nu - 2} (n-1) \left[-\nu v(r) + r v'(r) \right] + \mu^2 r^{-\nu} v(r).
\end{aligned}
\]

Group the terms:
\[
r^{-\nu - 2} \Big[ r^{2} v''(r) + (-2\nu + n - 1) r v'(r) + (\nu(\nu +1) - \nu(n-1)) v(r) \Big] + \mu^2 r^{-\nu} v(r) = 0.
\]

Since \(\nu = \frac{n-2}{2}\), evaluate coefficients:
\[
-2\nu + n - 1 = - (n - 2) + n - 1 = 1,
\]
\[
\nu(\nu +1) - \nu(n-1) = \nu^2 + \nu - \nu n + \nu = \nu^2 - \nu (n-2) = \frac{(n-2)^2}{4} - \frac{n-2}{2}(n-2) = -\frac{(n-2)(n-4)}{4}.
\]

Therefore, the equation reduces to:
\[
r^{-\nu - 2} \left[ r^{2} v''(r) + r v'(r) - \frac{(n-2)(n-4)}{4} v(r) \right] + \mu^{2} r^{-\nu} v(r) = 0.
\]

Multiply through by \(r^{\nu + 2}\):
\[
r^{2} v''(r) + r v'(r) + \left( \mu^{2} r^{2} - \frac{(n-2)(n-4)}{4} \right) v(r) = 0,
\]
which can be written as
\[
v''(r) + \frac{1}{r} v'(r) + \left( \mu^{2} - \frac{(n-2)(n-4)}{4 r^{2}} \right) v(r) = 0.
\]

This is the Bessel-type equation satisfied by \(v(r)\).



\subsection*{Step 4: Bessel-Type Form}

This equation is a Bessel-type differential equation. Recall the standard form of the Bessel equation:
\[
x^2 y'' + x y' + (x^2 - \nu^2) y = 0
\]

This can be rewritten as:
\[
y'' + \frac{1}{x} y' + \left(1 - \frac{\nu^2}{x^2} \right)y = 0
\]

Our transformed equation resembles this form, so the solution is:
\[
v(r) = J_\nu(\mu r)
\]

\subsection*{Step 5: Final Solution for $\phi$}

Thus, the solution in terms of the original function $\phi(r)$ is:
\[
\phi(r) = r^{-\nu} J_\nu(\mu r)
\]
\subsection{Connection to the 3D Radial Heat Equation}

When \(n = 3\), we have \(\nu = \frac{3 - 2}{2} = \frac{1}{2}\), and the coefficient of the potential term becomes:
\[
\frac{(n - 2)(n - 4)}{4} = \frac{(1)(-1)}{4} = -\frac{1}{4}.
\]

Thus, the Bessel-type equation reduces to:
\[
v''(r) + \frac{1}{r} v'(r) + \left( \mu^2 + \frac{1}{4 r^2} \right) v(r) = 0.
\]

However, in the time-dependent heat equation in three-dimensional spherical coordinates, we previously encountered the equation:
\[
\frac{\partial u}{\partial t} = \alpha \left( \frac{\partial^2 u}{\partial r^2} + \frac{2}{r} \frac{\partial u}{\partial r} \right),
\]
which, after applying separation of variables \(u(r,t) = \phi(r) T(t)\), leads to the radial ODE:
\[
\phi''(r) + \frac{2}{r} \phi'(r) + \lambda \phi(r) = 0,
\]
with solutions involving spherical Bessel functions when \(\phi(r)\) is assumed to be radial.

By applying the change of variable \(\phi(r) = \frac{v(r)}{r}\), we obtained:
\[
v''(r) + \left( \lambda - \frac{2}{r^2} \right) v(r) = 0,
\]
This shows that the classical 3D spherical case is a special case of the general \(n\)-dimensional radial problem, where the solution involves the Bessel function of order \(\nu = \frac{1}{2}\). In this case, the regular Bessel function \(J_{1/2}(\mu r)\) simplifies to a sine function:
\[
J_{1/2}(\mu r) = \sqrt{\frac{2}{\pi \mu r}} \sin(\mu r),
\]
so the radial solution becomes:
\[
\phi(r) = A \cdot \frac{\sin(\mu r)}{r},
\]
which is the spherical Bessel function of the first kind of order zero:
\[
j_0(\mu r) = \frac{\sin(\mu r)}{\mu r}.
\]


\section{Brownian Motion for 2D}
Recall the CDF:
\begin{equation*}\label{eqn:cdf_B_T}
u(T,r,R,\rho)= \mathbb{P}\left(\max_{t\in[0,T]}|B_t| < R \text{ and } |B_T| \le  \rho \Big||B_0|=r\right)
\end{equation*}
\textbf{Step 1: Expression for \(A_n\)}

\[
A_n = \frac{
\int_0^R r \cdot u(0, r) \cdot J_0\left( \frac{Z_n}{R} r \right) \, dr
}{
\int_0^R r \cdot \left[J_0\left( \frac{Z_n}{R} r \right)\right]^2 \, dr
}.
\]

Given the initial condition  \[
u(0, r) = f(r) = 
\begin{cases}
    1& r\le \rho\\
    0&r>\rho
\end{cases}
\]this becomes

\[
A_n = \frac{
\int_0^\rho r \cdot J_0\left( \frac{Z_n}{R} r \right) \, dr
}{
\int_0^R r \cdot \left[J_0\left( \frac{Z_n}{R} r \right)\right]^2 \, dr
}.
\]
\vspace{0.5em}
\textbf{Step 2: Use the known normalization result}
\[
\int_0^R r \left[J_0\left(\frac{Z_n}{R} r \right)\right]^2 dr = \frac{R^2}{2} \left[J_1(Z_n)\right]^2.
\]

Therefore,

\[
A_n = \frac{2}{R^2 [J_1(Z_n)]^2} \int_0^\rho r \, J_0\left( \frac{Z_n}{R} r \right) dr.
\]
\vspace{0.5em}
\textbf{Step 3: Evaluate the integral}

Recall the formula  (pg.46)

\[
\int x J_0(a x) dx = \frac{x}{a} J_1(a x) + C.
\]

\textbf{let} \(a=\frac{Z_n}{R}\) , \(x=\rho\)\\

Thus,

\[
\int_0^\rho r J_0\left(\frac{Z_n}{R} r\right) dr = \left. \frac{r R}{Z_n} J_1\left( \frac{Z_n}{R} r \right) \right|_0^\rho = \frac{\rho R}{Z_n} J_1\left( \frac{Z_n}{R} \rho \right).
\]
\vspace{0.5em}
\textbf{Step 4: Final formula for \(A_n\)}

Substituting the integral back:

\[
A_n = \frac{2 \rho}{R Z_n \left[J_1(Z_n)\right]^2} \, J_1\left( \frac{Z_n}{R} \rho \right).
\]
\textbf{Step 5: Final Solution for the CDF}

\[
u(T,r,R,\rho) = \sum_{n=1}^\infty \frac{2 \rho}{R Z_n [J_1(Z_n)]^2} J_1\left(\frac{Z_n}{R}\rho\right) J_0\left(\frac{Z_n}{R}r\right) e^{-\frac{Z_n^2}{2 R^2} T}
\]

\section{Brownian Motion for 3D}

The solution to the radially symmetric heat equation (in 3D with Dirichlet boundary at \( r = R \)) has the form:
\[
u(t, r) = \sum_{n=1}^{\infty} A_n \cdot e^{-\lambda_n t} \cdot \frac{\sin\left( \frac{n\pi r}{R} \right)}{r}
\]

At \( t = 0 \), this gives:
\[
f(r) = u(0, r) = \sum_{n=1}^\infty A_n \cdot \frac{\sin\left( \frac{n\pi r}{R} \right)}{r}
\]

Multiplying both sides by \( r \cdot \sin\left( \frac{m\pi r}{R} \right) \) and integrating over \( [0, R] \):
\[
\int_0^R f(r) \cdot r \cdot \sin\left( \frac{m\pi r}{R} \right) dr = A_m \cdot \int_0^R r\cdot\sin^2\left( \frac{m\pi r}{R} \right)  dr
\]

\subsection{Denominator (Norm)}

We recall the standard result:
\[
\int_0^R r\cdot\sin^2\left( \frac{n\pi r}{R} \right) dr = \frac{R^2}{2}
\]

\subsection{Numerator}

We compute the numerator using the definition of \( f(r) \):
\[
\int_0^R f(r)\cdot r\cdot \sin\left( \frac{n\pi r}{R} \right) dr
= \int_0^{\rho} r\cdot \sin\left( \frac{n\pi r}{R} \right) dr
\]

Make the substitution \( x = \frac{n\pi r}{R} \), so that:
\[
r = \frac{Rx}{n\pi}, \quad dr = \frac{R}{n\pi} dx
\]
When \( r = 0 \), \( x = 0 \), and when \( 0 \le r \le \rho \), \( x = \frac{n\pi r}{R} \)

Then:
\[
\int_0^{\rho} r\cdot \sin\left( \frac{n\pi r}{R} \right) dr
= \frac{R^2}{(n\pi)^2} \int_0^{\frac{n\pi \rho}{R}} x \cdot \sin(x) dx
\]

Using integration by parts: {\color{red}check it with it that use ax}
\[
\int x \sin\left(a x\right) \, dx = -\frac{x}{a}\cos \left(ax\right) + \frac{1}{a^2}\sin\left(a x\right) + C
\]

So:

\[
\int_0^{\rho} r \cdot \sin\left( \frac{n\pi r}{R} \right) dr
= 
-\frac{rR}{n\pi} \cos\left( \frac{n\pi r}{R} \right)
+ \frac{R^2}{(n\pi)^2}\sin\left( \frac{n\pi r}{R} \right)
\]
Then:
\[
\int_0^{\rho} r\cdot \sin\left( \frac{n\pi r}{R} \right) dr
= \left[
-\frac{rR}{n\pi} \cos\left( \frac{n\pi r}{R} \right) + \frac{R^2}{(n\pi)^2} \sin\left( \frac{n\pi r}{R} \right)
\right]_0^\rho
\]

\[
= -\frac{\rho R}{n\pi} \cos\left( \frac{n\pi \rho}{R} \right) + \frac{R^2}{(n\pi)^2} \sin\left( \frac{n\pi \rho}{R} \right)
\]
\subsection{Final Expression for \( A_n \)}

Dividing by the norm \( \frac{R^2}{2} \), we get:
\[
A_n = \frac{2}{R^2} 
\left[
-\frac{\rho R}{n\pi} \cos\left( \frac{n\pi \rho}{R} \right)
+ \frac{R^2}{(n\pi)^2}\sin\left( \frac{n\pi \rho}{R} \right)
\right]
\]

\newpage
\section{CDF of $B_T$}
\begin{lstlisting}[style = mypython]
# Brownian Motion in a Disk: CDF and Sampling
# This code computes the cumulative distribution function (CDF) of the radial distance
# of a Brownian motion conditioned to stay within a disk of radius R at time T.
# It uses a series expansion based on Bessel functions and performs inverse transform sampling
# to generate samples from this distribution.

import numpy as np
from scipy.special import j0, j1, jn_zeros
from scipy.interpolate import interp1d
from scipy.optimize import brentq
import matplotlib.pyplot as plt
import random

def compute_cdf(T, R, rho_vals, N_terms=100):
    """
    Compute CDF F(rho) = P(|B_T| <= rho | survive inside disk of radius R)
    using the truncated series expansion with N_terms.
    Parameters:
      T       : float, time horizon
      R       : float, disk radius
      rho_vals: array of floats, points where to compute CDF (0 to R)
      N_terms : int, number of terms in series expansion
    Returns:
      cdf_vals: array of floats, CDF values at rho_vals
    """
    alpha = jn_zeros(0, N_terms)  # First N_terms zeros of J0
    cdf_vals = np.zeros_like(rho_vals)
    r = 0.0  # Use the maximum rho for normalization
    for n in range(N_terms):
        Zn = alpha[n]
        # Precompute constants
        denom = R * Zn * (j1(Zn )**2)
        for i, rho in enumerate(rho_vals):
            if rho == 0:
                term = 0
            else:
                term = (2 * rho / denom) * j1(Zn * rho / R) * j0(Zn * r / R) * np.exp(- (Zn**2) * T / (2 * R**2))
            cdf_vals[i] += term
            
    # Ensure CDF starts at 0 and ends at 1 (numerical)
    if len(cdf_vals) > 0:
        cdf_vals -= np.min(cdf_vals)  # Normalize to start at 0
        cdf_vals /= np.max(cdf_vals)  # Normalize to end at 1
    return cdf_vals

def generate_sample_from_cdf(T, R, N_terms=50, grid_points=1000):
    """
    Generate a single sample from the distribution of |B_T| conditioned to stay inside disk radius R.
    """
    # Create fine grid of rho
    rho_grid = np.linspace(0, R, grid_points)
    cdf_vals = compute_cdf(T, R, rho_grid, N_terms)

    # Create interpolation function of the CDF
    cdf_interp = interp1d(rho_grid, cdf_vals, kind='linear', bounds_error=False, fill_value=(0,1))

    # Inverse transform sampling:
    u = random.uniform(0, 1)

     # Safety check: handle edges
    eps = 1e-10
    if u <= cdf_vals[0] + eps:
        return rho_grid[0]
    elif u >= cdf_vals[-1] - eps:
        return rho_grid[-1] - eps  # stay within interpolation range

    
    # Define function to find root for inversion: cdf(rho) - u = 0
    def root_func(rho):
        return cdf_interp(rho) - u

    # Find root in [0, R] using Brent's method
    # find p such thatf(p)=u
    # f(a) and f(b) must have opposite signs
    sample_rho = brentq(root_func, rho_grid[0], rho_grid[-1])
    # Return the sampled radial distance
    return sample_rho

# Example usage:
if __name__ == "__main__":
    T = 1.0
    R = 1.0
    print("Sampled values of |B_T|:")
    for _ in range(5):
        sample = generate_sample_from_cdf(T, R)
        print(f"{sample:.4f}")

    # Optional: plot the CDF for visualization
    rho_x = np.linspace(0, R, 500)
    cdf_y = compute_cdf(T, R, rho_x)
    plt.plot(rho_x, cdf_y, label='CDF of |B_T| conditioned on survival')
    plt.xlabel(r'$\rho$')
    plt.ylabel(r'CDF $F(\rho)$')
    plt.title(r'CDF of radial distance $|B_T|$ in disk of radius $R$')
    plt.grid(True)
    plt.legend()
    plt.show()

    """
    Output:
    Sampled values of |$B_T$|:
    0.4695
    0.2471
    0.4801
    0.5827
    0.5310
    """
\end{lstlisting}

\begin{appendix}
    \section{Generating sample paths using cdf}
    The cdf of a random variable $X$ is defined by $F_X(x)=\mathbb{P}(X\le x)$. If a cdf of a random variable $X$ is given, the inverse transform sampling can generate a sample of the random variable $X$. Inverse transform sampling 
\begin{algorithm}
\caption{Inverse transform sampling}
\label{alg:its}
\hspace*{\algorithmicindent}\textbf{Input:} $F_X(x)$ the cdf  of a function \\
\hspace*{\algorithmicindent}\textbf{Output:} A sample of random variable $X$
\begin{algorithmic}[1]
\Procedure{Inversion}{}
\State Find and save the inverse of cdf, $F_X^{-1}:(0,1)\to \mathbb{R}$ 
\EndProcedure
\Procedure{Sampling}{}
\State Generate a random number $u$ uniformly on $(0,1)$
\State \Return $F_X^{-1}(u)$
\EndProcedure
\end{algorithmic}
\end{algorithm}

    
    Recall from \eqref{eqn:cdf_B_T} that the cdf of $|B_T|$ is given by 
\[
u(T,0,R,\rho)= \mathbb{P}\left(\max_{t\in[0,T]}|B_t| < R \text{ and } |B_T| \tikzmark{cdf}\le  \rho \Big||B_0|=0\right)
\]
Fixing $T$ and $R$, we evaluate $F(\rho):=u(T,0,\rho)$ using Bessel functions. Then, we evaluate the inverse $F^{-1}(u)$. This inverse function needs to be saved for many different values of $u$, $T$, and $R$ in a format that can be queried upon request in lightspeed. Using samples of uniform random variable, we can generate sample of $B_T$.
\end{appendix}


\end{document}
